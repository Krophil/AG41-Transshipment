\documentclass[a4paper,12pt]{article}
\usepackage[utf8]{inputenc}
\usepackage[top=2cm, bottom=2cm, left=2cm, right=2cm]{geometry}
\usepackage{changepage}
\usepackage[frenchb]{babel} % adopte les règles de typographie française
\usepackage{libertine}
\usepackage[T1]{fontenc}
\usepackage{amsmath}
\usepackage{amssymb}
\usepackage{graphicx}
\usepackage{tikz}
\usetikzlibrary{shapes.geometric}
\usepackage{float}
\usepackage{listings}
\renewcommand{\familydefault}{\sfdefault}
\renewcommand{\Frlabelitemi}{$\bullet$}
\renewcommand{\Frlabelitemii}{$\circ$}
\renewcommand{\Frlabelitemiii}{*}


\begin{document}

\begin{titlepage}
    \centering
    \vspace*{\fill}
	{\huge\bfseries AG41\\[0.3cm]}
	{\LARGE\bfseries PROJET D'OPTIMISATION EXACTE\\}
	{\LARGE\bfseries Problème de transbordement\\[0.5cm]}
	{\Large\itshape Esia Belbachir, Pierre Brunet de Monthélie\\[0.5cm]}
    {\large P2016\\[1cm]}
\vspace*{\fill}
\end{titlepage}

\newpage
\tableofcontents
\listoffigures
\newpage

\section{Analyse mathématique}

\subsection{Paramètres}

\begin{description}
 \item[$V$] Ensemble des noeuds du graphe tel que $V = C \cup F \cup P$
  \item[$C$] Ensemble des noeuds clients tel que $\forall i \in C, b_i > 0$
 \item[$P$] Ensemble des noeuds plateformes tel que $\forall i \in P, b_i = 0$
 \item[$F$] Ensemble des noeuds fournisseurs tel que $\forall i \in F, b_i < 0$
 \item[$A$] Ensemble des arcs orientés $e = (i,j) \in V^2$ tel que tous les fournisseurs ont un arcs vers toutes les plateformes et toutes les plateformes ont un arc vers tous les clients :\\
 $\forall (i,j) \in A, \mbox{ alors } (i \in C \mbox{ et }j \in P) \mbox{ ou }(i \in P\mbox{ et }j \in C)$
 \item[$b_i$] Quantité de produits demandée par le noeud $i \in V$ \\ $\sum\limits_{i \in V} b_i = 0$
 \item[$g_i$] Coût de transbordement unitaire de la plateforme $i \in P$
 \item[$s_i$] Temps de transbordement de la plateforme $i \in P$
 \item[$u_{ij}$] Capacité de l'arc $(i,j) \in A$
 \item[$c_{ij}$] Coût fixe d'utilisation de l'arc $(i,j) \in A$ si cet arc transporte des produits
 \item[$h_{ij}$] Coût unitaire de transport de l'arc $(i,j) \in A$
 \item[$t_{ij}$] Temps de transport de l'arc $(i,j) \in A$
 \item[$T$] Délai maximum pour acheminer tous les produits
\end{description}

\subsubsection{Remarques}

\begin{itemize}
 \item $C\cap F\cap C = \varnothing$
 \item $\forall i \in V, b_i$ est égale à la différence entre la quantité de produits qui doit rentrer et la quantité de produits qui doit sortir du noeud $i$
\end{itemize}


\subsection{Variables}

\begin{description}
 \item[$x_{ij}$] Quantité de produits transportés sur l'arc $(i,j) \in A$
 \item[$x_{ijk}$] Quantité de produits transportés sur l'arc $(i,j)$ puis sur l'arc $(j,k)$ avec $i \in F$, $j \in P$ et $k \in C$ \\
 $(i,j,k)$ représente le chemin emprunté par un paquet de $x$ produits
 \item[$y_{ij}$] Valeur binaire égale à  $1$ si l'arc $(i,j) \in A$ est utilisé pour transporter des produits, $0$ sinon
 \item[$y_{ijk}$] Valeur binaire égale à  $1$ si le chemin $(i,j,k) \in F\times P \times C$ est utilisé pour transporter un paquet de produits, $0$ sinon \\
 Si $(i,j)$ et $(j,k)$ sont utilisés, mais pas pour le même paquet, $y_{ijk}$ est égal à $0$
\end{description}

\subsection{Fonction objectif}

$\min z = \sum \limits_{(i,j) \in F\times P} y_{ij} ( c_{ij} + x_{ij} (h_{ij} + g_i)) + \sum \limits_{(i,j) \in P \times C} y_{ij} (x_{ij} h _{ij}  + c_{ij})$

\subsection{Contraintes}


\begin{tabular}{r l}
 Temps & $\forall (i,j,k) \in (F\times P\times C), y_{ijk}(t_{ij} + t_{jk} + s_i) \leqslant T$ \\
 Capacité & $\forall (i,j) \in A, x_{ij} \leqslant u_{ij}$ \\
 Conservation & $\forall j \in V, \sum \limits_{\substack {k \in V \\ (j,k) \in A}} x_{jk} + \sum \limits_{\substack {i \in V \\ (i,j) \in A}} x_{ij} = b_i$ \\
 Relation variables & $\forall (i,j) \in F\times P, x_{ij} = \sum\limits_{k \in C} x_{ijk}$\\
  &$\forall (j,k) \in F\times P, x_{jk} = \sum\limits_{i \in F} x_{ijk}$\\
  &$\forall (i,j) \in A \mbox{ si } x_{ij} > 0, \mbox{ alors } y_{ij} = 1$ \\
  &$\forall (i,j,k) \in F \times P \times C, \mbox{ si } x_{ijk} > 0, \mbox{ alors } y_{ijk} = y_{ij} = y_{jk} = 1$
 
\end{tabular}

\subsection{Représentation graphique}

\begin{figure}[H]
    \begin{center}
        \begin{tikzpicture}[scale=1.5]
        \tikzstyle{every node}=[ellipse, draw]
        \foreach \i in {1,...,3}
        {
            \path (2,\i*2) node (v\i) {Dépôt};
            \path (4,\i*2) node (w\i) {Plateforme};
            \path (6,\i*2) node (x\i) {Client};
        }
            
        \foreach \i in {1,...,3}
        {
            \foreach \j in {1,...,3}
            {
                \draw (v\i) -- (w\j);
                \draw (w\i) -- (x\j);
            }
        }
        \end{tikzpicture}
    \caption{Représentation du problème}
    \end{center}
\end{figure}

\section{Questionnement et algorithmes}

En mettant en évidence les contraintes et les variables, nous avons identifié des caractéristiques permettant de fractionner et de simplifier le problème considéré.


\subsection{Un problème de temps}
Dans notre analyse mathématique, nous avons inclu la contrainte temporelle de la manière suivant : tout paquet de taille $y$ passant par les noeuds $i$,$j$ et $k$ doit être acheminé dans un temps inférieur à une limite $T$. Afin de remplir cette contrainte, nous avons adapté la représentation graphique de ce problème de la manière suivante : chaque noeud plateforme est remplacé par un graphe biparti entre les arcs entrants et les arcs sortants. 


\begin{figure}[H]
    \begin{center}
        \begin{tikzpicture}[scale=2]
        \tikzstyle{every node}=[ellipse, draw];
            \foreach \i in {1,...,3}
            {
                \path (2,\i) node (v\i) {$e_\i$};
                \path (4,\i) node (w\i) {$s_\i$};
            }
                
            \foreach \i in {1,...,3}
            {
                \foreach \j in {1,...,3}
                {
                    \draw (v\i) -- (w\j);
                }
            }
            \foreach \j in {1,...,3}
            {
                \draw [dashed] (0, \j) -- (v\j);
                \draw [dashed] (w\j) -- (6, \j);
            }
            \fill[white] (0,0.9) rectangle (0.75,3.1);
            \fill[white] (5.5,0.9) rectangle (6.5,3.1);
        \end{tikzpicture}
    \end{center}
    \caption{Décomposition d'une plateforme avec 3 dépôts et 3 clients}
\end{figure}
 
Pour chaque noeud plateforme $i$ :
\begin{itemize}
    \item Caractéristique des noeuds créés $j$ :
        \begin{itemize}
            \item $b_j = 0$
        \end{itemize}
    \item Caractéristiques des arcs créés $(j,k)$ :
        \begin{itemize}
            \item $t_{jk} = s_i$
            \item $h_{jk} = g_i$
            \item $c_{jk} = 0$
            \item Si le chemin d'un fournisseur à un client enpruntant l'arc $(j,k)$ a un temps total de transport supérieur à $T$, $u_{jk} = 0$ \\
            Sinon, $u_{jk} = + \infty$
        \end{itemize}
\end{itemize}

En représentant le problème ainsi, on peut ``ignorer'' la contrainte de temps puisqu'elle est intégrée dans la structure même du graphe.


\subsection{Un problème de flot maximal à coût minimum}

Une fois la contrainte de temps mise de côté, on constate que le problème proposé est en fait un problème de flux maximal à coût minimum.

\subsubsection{Flot maximal}

En ajoutant un noeud source et un noeud cible au graphe, on se retrouve face à un problème de flux maximal classique.

\begin{figure}[H]
    \begin{center}
        \begin{tikzpicture}[scale=1.5]
            \tikzstyle{every node}=[ellipse, draw];
            \path[red] (0, 4) node (s) {source};
            \path[red] (8, 4) node (t) {cible};
            \foreach \i in {1,...,3}
            {
                \path (2,\i*2) node (v\i) {Dépôt};
                \path (4,\i*2) node (w\i) {Plateforme};
                \path (6,\i*2) node (x\i) {Client};
            }
                
            \foreach \i in {1,...,3}
            {
                \draw[red] (s) -- (v\i);
                \draw[red] (x\i) -- (t);
                \foreach \j in {1,...,3}
                {
                    \draw (v\i) -- (w\j);
                    \draw (w\i) -- (x\j);
                }
            }
        \end{tikzpicture}
    \end{center}
    \caption{Ajout de la source et de la cible sur le graphe pour faciliter le traitement}
\end{figure}

Pour résoudre ce problème, nous pouvons utiliser l'algorithme de Edmonds-Karp, implémentation de la méthode de Ford-Fulkerson. Il repose sur la recherche d'un chemin d'amélioration dans le graph résiduel.

Soient $c(u,v)$ la capacité de l'arc $(u,v)$ et $f(u,v)$ le flot de $(u,v)$.

\begin{verbatim}
dfs(G, u)
DEBUT
    POUR u dans V
        Marquer u
        POUR tous les noeuds v dans noeudsAdjacents(u)
            SI v n'est pas marqué
                dfs(G, v)
            FIN SI
        FIN POUR
    FIN POUR
FIN
\end{verbatim}

\begin{verbatim}
fordFulkerson(G)
DEBUT
    POUR (u,v) dans Arcs(G)
        f(u,v) = 0
        Marquer s
        TANT QUE il existe une chaîne améliorante µ
            fluxDir = min(c(u,v)-f(u,v)), 
                        où u est un arc direct de µ
            fluxInd = min(f(u,v)),
                        où u est un arc indirect de µ
            flux = min(fluxDir, fluxInd)
            POUR (u,v) dans ArcsDirects(µ)
                f(u,v) = f(u,v) + flux
            FIN POUR
            POUR (u,v) dans ArcsIndirects(µ)
                f(u,v) = f(u,v) - flux
            FIN POUR
        FIN TANT QUE
    FIN POUR
FIN
\end{verbatim}


La complexité est égale à $m^2*n$ avec $m$ le nombre d'arcs et $n$ le nombre de noeuds.

\subsubsection{Coût minimum}

Le problème de flux à coût minimum présenté ici est plus complexe que les problèmes de flux à coût minimum classiques car les arcs possèdent dans notre cas un coût unitaire et un coût fixe.

Nous avons alors décidé de calculer le coût unitaire à partir de la moyenne des coûts unitaire et fixe selon la capacité de l'arc.

Par exemple, pour une capacité de $n$ produits, on aurait :
\begin{equation*}
    \frac{c_{moyen} = n*c_{unitaire}+c_{fixe}}{n}
\end{equation*}

Une solution usuelle d'un problème de flux à coût minimum est de chercher un cycle négatif dans le graphe résiduel des coûts. En utilisant les coûts moyens comme coûts unitaires, on peut utiliser cette méthode.

Une autre possibilité serait de résoudre le problème avec, tout d'abord, les coûts unitaires. On aurait alors une solution basé sur la quantité de produits. Ensuite, on améliorerait cette solution avec les coûts fixes.

\begin{lstlisting}[
mathescape,
 columns=fullflexible,
 basicstyle=\fontfamily{lmvtt}\selectfont,
 escapeinside=::,
 ]
DEBUT
    Initialement, $\phi = (0, ..., 0) ; G_{\phi}^{e} = R$
    FAIRE
        trouver C un chemin de co:\^u:t minimal de s :à: p dans le graphe d'
        $\delta = min (r_{ij}) (avec r_{ij} = c_{ij} - \phi_{ij} )$
            $(i,j) \in C$
        POUR tout arc u = (i, j) de C FAIRE
            SI (i, j) arc du graphe
                FAIRE $\phi (u) = \phi (u) + \delta$
            SINON
                SI (j, i) arc du graphe, $\phi (u) = \phi (u) - \delta$
                    modifier le graphe d':é:cart $G_\phi^e$
                FIN SI
            FIN SINON
            FIN SI
        FIN POUR
    TANT QUE il existe un chemin de s :à: p dans $G_\phi^e$
FIN
\end{lstlisting}

\section{Résumé}

Afin de résoudre le problème proposé, nous avons segmenté sa résolution en 3 sous-problèmes : temps, flot  maximal et coût minimum. 

Les 2 premiers sont nécessaires à l'obtention d'une solution qui remplit les contraintes, le dernier permet d'améliorer la solution générée. 

Pour trouver une solution, le temps minimum de calcul est donc celui de la construction du graphe et de la résolution du problème de flot maximal selon les méthodes expliquées plus haut. 

Ensuite, plus on laissera le programme s'exécuter longtemps, plus la solution sera bonne, jusqu'à atteindre la solution optimale.

La complexité de ce programme est égale à la somme des complexités des différents algorithmes utilisés.

\section{Implémentation}

Pour pouvoir impl\'ementer notre algorithme, nous avons d\'ecid\'e de cr\'eer notre propre structure de graphe afin qu'elle s'adapte aux diff\'erentes m\'ethodes que nous avons utilisées. La s\'eparation des plateformes a ainsi \'et\'e directement prise en compte dans la structure.

Le programme est cod\'e en Java.

\subsection{La classe Graph}

\subsubsection{Graph : une classe g\'en\'erique}

La classe Graph est une classe g\'en\'erique, ce qui permet une certaine flexibilité quant aux données portées par les noeuds et les arcs. Elle permet de stocker la structure du graphe indépendament des données. Le Graph fonctionne avec des numéros uniques identifiant les noeuds. Les arcs sont également identifiés en utilisant un couple d'identifiants.

Dans notre solver, nous n'avons utilisé que les classes Node et Edge en paramètre de classe de Graph, mais on pourait tout à fait utiliser d'autres couples <node type, edge type> pour améliorer les algorithmes.

\subsubsection{Les classes Node et les Edge}

Les classes Node et Edge sont composées des données du problème en terme d'informations portées par les noeuds et les arcs.

La classe Node se compose donc de ces données de base, mais aussi d'un attribut parent qui est utilis\'e dans le DFS permettant de g\'en\'erer tous les cycles du graphes. Il est utilis\'e pour remonter jusqu'au parent cr\'eant le cycle.

\subsection{La lecture du fichier de problème}

La lecture du fichier de problème est la 1er étape de l'algorithme. Elle permet de construire le graphe de base du problème. Elle permet aussi d'éliminer les arcs dont la capacité est nulle et de déterminer les 3 ensembles de noeuds que sont les fournisseurs, les plateformes et les clients. C'est également à cette étape qu'est détecté si le nombre de produits offerts par les founisseurs correspond au nombre de produits demandé par les clients.

\subsection{La s\'eparation des plateformes}

La s\'eparation des plateformes permet de prendre en compte la contrainte de temps. Pour la mettre en place en Java, on utilise la liste de noeuds plateformes crée à la lecture du fichier de problème.

Pour chaque plateforme, on crée de nouveaux noeuds comme nous l'avions prévu dans la phase d'analyse du problème de transbordement. Les noeuds initiaux sont supprimés à l'issu de cette étape, et 2 listes sont crées : la liste des plateformes gauches (liées aux fournisseurs) et la liste des plateformes droites (liées aux clients).

Une fois le graphe modifié, on vérifie que tous les chemins respectent la contrainte de temps total de transport. Si ce n'est pas le cas, l'arc entre la plateforme gauche et la plateforme droite inclue dans le chemin est supprimé.

Puisqu'on supprime les plateformes d'origine, il est impossible de savoir à quel plateforme d'origine correspond une plateforme gauche ou une plateforme droite. Il est donc nécessaire de créer une table associative sous la forme d'une HashMap qui à chaque nouvelle plateforme associe l'ancienne plateforme.

\subsection{La cr\'eation d'une solution initiale}

La génération d'une solution initial s'effectue en 2 étapes.

D'abord le remplissage arbitraires d'arcs jusqu'à leur maximum. Pour cela, tous les chemins fournisseur-plateforme gauche-plateforme droite-client sont remplis au maximum jusqu'à ce que toute la demande soit remplie, ou bien jusqu'à ce qu'on arrive à la fin des chemins.

Si toute la demande n'a pas pu être totalement remplie, on ajoute une source et une cible au graphe et un DFS est lancé sur le graphe résiduel pour trouver des chemins d'amélioration. Comme la taille du problème est variable, on ne recherche pas le meilleur chemin d'amélioration, mais le 1er que l'on trouve, afin d'éviter de tout parcourir. Pour la même raison, nous avons décidé d'implémenter un DFS procédural et non récursif pour éviter un overflow de la pile. 

Si on n'arrive pas à un flot permettant de répondre à toute la demande, alors le problème est insoluble.

\subsection{Un diagramme de flot maximal \`a co\^ut maximal}

Pour cet algorithme, nous n'avons pas besoin de source et de cible car l'algorithme circule \`a partir de l'ensemble des d\'ep\^ots. De plus, la source et la cible sont d\'esormais inutile car, leur capacit\'e \'etant \'egale au d\'ep\^ot ou au client qui leur est associ\'e, ils sont d\'ej\`a charg\'e au maximum. On enl\`eve donc ces deux noeuds \`a la fin de l'algorithme pr\'ec\'edent.

\`A partir de cette solution, on va chercher \`a l'am\'eliorer en cherchant les cycles n\'egatifs comme nous avions d\'ecid\'e de le faire dans la premi\`ere partie. En revanche, l'ajout unitaire de produits, pour chaque cycle, risque d'\^etre tr\`es lent car certains arcs ont une capicit\'e tr\`es importante avec une demande aussi importante.

Nous avons donc d\'ecid\'e de r\'ecup\'erer la capacit\'e maximale du cycle et de calculer le graphe r\'esiduel correspondant. \`A chaque it\'eration, l'algorithme v\'erifie qu'il y a des cycles n\'egatifs et que le temps maximal de calcul n'est pas atteint.

Lorsqu'un co\^ut est n\'egatif, le cycle est consid\'er\'e comme n\'egatif. On ajoute et soustraie alors les diff\'erents changements de flot de produits. Puis on recommence une it\'eration si il reste du temps.

\subsubsection{La recherche de cycles}

Pour trouver tous les cycles d'un graphe, plusieurs m\'ethodes sont possibles. Nous avons d\'ecid\'e d'utiliser le DFS car c'est la m\'ethode nous semblant la plus logique et la plus concr\`ete.

On lance un DFS r\'ecursif, on cr\'ee une liste de noeuds candidats et une liste de noeuds d\'ecouverts. Si un noeud est trouv\'e une deuxi\`eme fois et s'il fait partie de la liste des candidats, on remonte jusqu'\`a ce qu'on trouve un noeud parent \'egale \`a ce m\^eme noeud.

Ensuite, on enl\`eve tous les cycles qui sont trouv\'es plusieurs fois \`a cause du d\'ecalage. Enfin, on inverse l'ordre de chaque cycle.

Cette impl\'ementation \'etant en r\'ecursif, elle est tr\`es puissante mais tr\`es gourmande en ressource. Ainsi, pour les graphes de 20 noeuds et plus, le DFS met \'enorm\'ement de temps \`a se faire. Le temps d'une it\'eration est donc grandement augment\'ee.

\subsubsection{La prise en compte des co\^uts fixes}

Le principe du co\^ut fixe est l'ajout d'un co\^ut suppl\'ementaire d'un arc d\`es que ce dernier est utilis\'e, m\^eme pour un produit. Notre but est donc de minimiser le nombre d'arcs utilis\'es afin d'\'eviter des co\^uts trop importants.

Pour prendre compte les co\^ut fixes, il suffit d'ajouter, dans le calcul du co\^ut d'un cycle, une condition afin d'ajouter ou de soustraire au co\^ut total le prix fixe de l'arc.

\subsection{Le temps de calcul}

Comme on ne peut pas laisser tourner l'algorithme indéfiniement, un temps maximum de calcul est donné en paramètre au programme. De plus, pour plus de précision, nous avons indiqué le temps que chaque étape prend pour s'effectuer : la lecture de fichier, la création d'une solution initial (qui prend en compte la séparation des plateformes), l'amélioration de la solution.

Quel que soit le temps imparti, les 2 premières étapes sont toujours remplies afin que le programme retourne toujours au moins une solution. Si il reste du temps, on améliore la solution initiale. Comme pour la lecture de fichier et l'initialisation, une itération d'amélioration ne peut pas être interrompue. Ainsi, le temps total n'est pas forcément égal au temps de calcul imparti.

\section{R\'esultats}

Après avoir mené les tests dont les résultats sont indiqués ci-dessous, nous nous sommes rendus  compte que les algorithmes de solution initial et d'amélioration étaient très longs. Le dédoublement des plateformes augmente en effet énormément la taille du graphe et les DFS sont donc très long. Changer d'algorithme ou trouver un moyen de ne pas dédoubler les plateformes seraient des moyens de palier à ce problème.

\begin{center}
 \begin{tabular}{r | c | l}
\textbf{Fichier} & \textbf{R\'esultat} & \textbf{Temps} \\ \hline
transshipment1.txt & 212.0 & 5.397s \\
transshipment2.txt & pb dans la lecture du fichier & \\
tshp006-01.txt & pb dans la lecture du fichier & \\
tshp10-01.txt & 3940.5911509852026 & 0.02s \\
tshp10-02.txt & pb dans la lecture du fichier & \\
tshp010-01.txt & 4756.506448083509 & 157.089s \\
tshp010-02.txt & 2126.9661327456147 & 1.856s \\
tshp010-03.txt & 1951.8313355932821 & 0.079s \\
tshp010-04.txt & 2114.3288184684357 & 0.888s \\
tshp010-05.txt & 2866.751823637155 & 1.253s \\
tshp020-01.txt & 4573.544241371602 & stopp\'e au bout de 20 min \\
tshp020-02.txt & 7128.861258575862 & stopp\'e au bout de 35 min \\
tshp020-03.txt & 9662.086844423093 & stopp\'e au bout de 60 min \\
tshp020-04.txt & 11159.946168444236 & stopp\'e au bout de 45 min \\
tshp020-05.txt & 11683.402761321786 &  \\
tshp050-01.txt & 18267.737819422495 & stopp\'e au bout de 45 min (Attention : la demande \\
& & n'est pas statisfaite : manquent 14 produits/230) \\
tshp050-02.txt & 17314.81989053033 & stopp\'e au bout de 30 min (Attention : la demande \\
& & n'est pas statisfaite : manquent 27 produits/218) \\
tshp050-03.txt & 10176.322729872461 & stopp\'e au bout de 30 min \\
tshp050-04.txt & 21131.564098532177 & stopp\'e au bout de 20 min \\
tshp050-05.txt & 19060.701180329823 & stopp\'e au bout de 20 min \\
tshp100-01.txt & 45838.598997389054 & stopp\'e au bout de 30 min (Attention : la demande \\
& & n'est pas statisfaite : manquent 41 produits/488) \\
tshp100-02.txt & 47376.39000754389 & stopp\'e au bout de 20 min (Attention : la demande \\
& & n'est pas statisfaite : manquent 4 produits/497) \\
tshp100-03.txt & 35390.43760148906 & stopp\'e au bout de 20 min (Attention : la demande \\
& & n'est pas statisfaite : manquent 103 produits/460) \\
tshp100-04.txt & 19945.698281503657 & La demande n'est pas statisfaite : manquent \\
& & 119 produits/498 : infaisable \\
tshp100-05.txt & 25741.323007876163 & La demande n'est pas statisfaite : manquent \\
& & 121 produits/474 : infaisable \\
tshp500-01.txt & 212217.19965555932 & manquent 280 produits/2506 \\
tshp500-02.txt & \o{} & Infaisable \\
tshp500-03.txt & \o{} & Infaisable \\
tshp500-04.txt & \o{} & Infaisable \\
tshp500-05.txt & \o{} & Infaisable \\
  
 \end{tabular}

\end{center}

\end{document}